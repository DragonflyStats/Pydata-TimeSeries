Gambler's Fallacy


M=100000

Lands=ceiling(runif(M)*2)

# 1 for Red
# 2 for Black
chain=1

for(i in 2:M)
{
if(Lands[i]==Lands[i-1])
  {
  chain[i]=chain[i-1]+1 
  }else{chain[i]=1}

}



\newpage


%---------------------------------------------------------------%
\section{Gambler's Ruin}

Consider a gambler who starts with an initial fortune of $1 and then on each successive gamble
either wins \$1 or loses \$1 independent of the past with probabilities p and q = 1-p respectively.
Suppose the gambler has a starting kitty of A. 
This gamblers places bets with the “Banker”, who has an initial fortune B. We will look at the game from the perspective of the gambler only.
The Banker is, by convention, the richer of the two.
\begin{itemize}
\item Probability of successful gamble for gambler : p
\item Probability of unsuccessful gamble for gambler : q 	(where q =  1 - p )
\item Ratio of success probability to failure success:	$s = p / q$
\item Conventionally the game is biased in favour of the Banker (i.e. $q>p$ and $s<1$)
\end{itemize}
Let $R_n$ denote the Gambler’s total fortune after the $n-$th gamble.
If the Gambler wins the first game, his wealth becomes $R_n =A+1$.
If he loses the first gamble, his wealth becomes $R_n = A-1$.
The entire sum of money at stake is the “Jackpot” i.e.   $A+B$.
The game ends when the Gambler wins the Jackpot ($R_n = A+B$) or loses everything ($R_n = 0$).
\subsection{Simulation a Single Gamble}
To simulate one single bet, compute a single random number between 0 and 1.
\begin{framed}
\begin{verbatim}
runif(1)
\end{verbatim}
\end{framed}
Lets assume that the game is biased in favour of the Banker
p = 0.45 , q = 0.55.
If the number is less than 0.45, the gamble wins. Otherwise the Banker wins.
\begin{verbatim}
> runif(1)
[1] 0.1251274
>#Gambler Loses
>
> runif(1)
[1] 0.754075
>#Gambler wins
>
> runif(1)
[1] 0.2132148
>#Gambler Loses
>
> runif(1)
[1] 0.8306269
\end{verbatim}
%----------------------------------------------------------------------%
Rn=c(20)
A=20;B=100;p=0.47
 
 
bet = runif(1)
if (bet < 0.40)
{
A = A+1; B =B-1
}else{A=A-1;B=B+1}
#Save the values from each bet
R=c(R,A)
Lets put this in a loop (300 iterations should be enough)
 
For ( I =1:300)
{
bet = runif(1)
if (bet < p)
{
A = A+1; B =B-1
}else{A=A-1;B=B+1}
#Save the values from each bet
R=c(R,A)
}
Consider a gambler who starts with an initial fortune of $1 and then on each successive gamble either wins $1 or loses $1 independent of the past with probabilities p and q = 1-p respectively.

Let Rn denote the total fortune after the nth gamble.

Game:  coin toss 
Gambler wins if coin throw is “heads”
	Banker wins if coint throw is “tails”. 
Probability of success (heads) : p
Probability of failure (tails) : q 	(where q =  1 - p )

Ratio of success probability to failure success:	s = p / q

Game is over if Gambler wins the jackpot
Gamler loses everything

The gambler’s objective is to reach a total fortune of $N, without first getting ruined (running out of money). 
If the gambler succeeds,then the gambler is said to win the game. In any case, the gambler stops playing after winning or getting ruined, whichever happens first. 
%--------------------------------------------------------------------------------%
\begin{framed}
\begin{verbatim}
Kitty=300
p=0.45
#########################
q=1-p
while(Kitty >0)
  {
  ProbVal=runif(1)
  if(ProbVal <= p)
     {
     Kitty=Kitty+1
     }else{Kitty=Kitty-1}
  Balance=c(Balance,Kitty)
  }
\begin{verbatim}
\begin{framed}
%--------------------------------------------------------------------------------%
\begin{framed}
\begin{verbatim}
Kitty=100
p=0.47
q=1-p
Balance=Kitty
#########################
while(Kitty >0)
  {
  ProbVal=runif(1)
  if(ProbVal <= p)
     {
     Kitty=Kitty+1
     }else{Kitty=Kitty-1}
  Balance=c(Balance,Kitty)
  }
Balance
length(Balance)
plot(Balance,type="l",col="red")
\begin{verbatim}
\begin{framed}
%--------------------------------------------------------------------------------%
%--------------------------------------------------------------------------------%


A state diagram for a simple example is shown in the figure on the right, using a directed graph to picture the state transitions. The states represent whether a hypothetical stock market is exhibiting a bull market, bear market, or stagnant market trend during a given week. According to the figure, a bull week is followed by another bull week 90\% of the time, a bear week 7.5\% of the time, and a stagnant week the other 2.5\% of the time. Labelling the state space {1 = bull, 2 = bear, 3 = stagnant} the transition matrix for this example is

P = \begin{bmatrix}
0.9 & 0.075 & 0.025 \\
0.15 & 0.8 & 0.05 \\
0.25 & 0.25 & 0.5
\end{bmatrix}.

%----------------------------------------------------%


The distribution over states can be written as a stochastic row vector x with the relation x(n + 1) = x(n)P. So if at time n the system is in state x(n), then three time periods later, at time n + 3 the distribution is

\begin{align}
x^{(n+3)} &= x^{(n+2)} P = \left(x^{(n+1)} P\right) P \\\\
   &= x^{(n+1)} P^2 = \left( x^{(n)} P \right) P^2\\
   &= x^{(n)} P^3 \\
\end{align}
%----------------------------------------------------%

In particular, if at time n the system is in state 2 (bear), then at time n + 3 the distribution is

\begin{align}
x^{(n+3)} &= \begin{bmatrix} 0 & 1 & 0 \end{bmatrix} 
\begin{bmatrix}
0.9 & 0.075 & 0.025 \\
0.15 & 0.8 & 0.05 \\
0.25 & 0.25 & 0.5
\end{bmatrix}^3 \\
   &= \begin{bmatrix} 0 & 1 & 0 \end{bmatrix} \begin{bmatrix}
 0.7745 & 0.17875 & 0.04675 \\
 0.3575 & 0.56825 & 0.07425 \\
 0.4675 & 0.37125 & 0.16125 \\
\end{bmatrix} \\
& = \begin{bmatrix} 0.3575 & 0.56825 & 0.07425 \end{bmatrix}.
\end{align}

%----------------------------------------------------%
Using the transition matrix it is possible to calculate, for example, the long-term fraction of weeks during which the market is stagnant, or the average number of weeks it will take to go from a stagnant to a bull market. Using the transition probabilities, the steady-state probabilities indicate that 62.5\% of weeks will be in a bull market, 31.25\% of weeks will be in a bear market and 6.25\% of weeks will be stagnant, since:

\[\lim_{N\to \infty } \, P^N=
\begin{bmatrix}
 0.625 & 0.3125 & 0.0625 \\
 0.625 & 0.3125 & 0.0625 \\
 0.625 & 0.3125 & 0.0625 \\
\end{bmatrix}\]

A thorough development and many examples can be found in the on-line monograph Meyn & Tweedie 2005.[6]

A finite state machine can be used as a representation of a Markov chain. Assuming a sequence of independent and identically distributed input signals (for example, symbols from a binary alphabet chosen by coin tosses), if the machine is in state y at time n, then the probability that it moves to state x at time n + 1 depends only on the current state.

Transient evolution


\end{document}